\begin{abstract}
Given a graph $G = (V, E)$, a coloring function $\chi: V \rightarrow
C$, assigning each vertex a color, is called \emph{convex} if, for
every color $c \in C$, the set of vertices with color $c$ induces a
connected subgraph of $G$.
%
In the \textsc{Convex Recoloring} problem a colored graph $G_\chi$ is
given, and the goal is to find a convex coloring $\chi'$ of $G$ that
\emph{recolors} a minimum number of vertices.
%
The \textsc{2-Convex Recoloring} problem (\TWOCR{}) is the special
case, where the given coloring $\chi$ assigns the same color to at
most two vertices.  \TWOCR{} is known to be NP-hard even if $G$ is a
path.

We show that weighted \TWOCR{} problem cannot be approximated within any ratio, unless P$=$NP.
%
On the other hand, we provide an alternative definition of
(unweighted) \TWOCR{} in terms of maximum independent set of paths,
which leads to a natural greedy algorithm.  We prove that its
approximation ratio is $\frac{3}{2}$ and show that this analysis is
tight.
%
This is the first constant factor approximation algorithm for a
variant of \CRP{} in general graphs.
%
For the special case, where $G$ is a path, the algorithm obtains a
ratio of $\frac{5}{4}$, an improvement over the previous best known
approximation.
%
We also consider the problem of determining whether a given graph has
a convex recoloring of size $k$.  We use the above mentioned
characterization of \TWOCR{} to show that a problem kernel of size
$4k$ can be obtained in linear time and to design a $O(|E|) + 2^{O(k
  \log k)}$ time algorithm for parametrized \TWOCR{}.
\end{abstract}
