
\section{Greedy Algorithm}
\label{sec:greedy}

In this section we describe a natural greedy algorithm to construct a maximal
independent set of colorable paths, 
we discuss how this algorithm can be implemented,
and define the corresponding path-recoloring it computes.

%%%%%

\subsection{The algorithm}
%
While we are not attempting to achieve an approximation to the size
of the maximum independent set of colorable paths,
the alternative definition given at the end of the previous section
leads us to a natural greedy algorithm: 
choose the shortest colorable path that 
is not in conflict with colorable paths already been chosen and add it
to an independent set of colorable paths.
%
A formal description of this algorithm is given in
Algorithm~\ref{alg:conv:greedy}.

\begin{algorithm}
\begin{algorithmic}

\State $I \gets \emptyset$

\While{there is a colorable path, not in conflict with $I$}
\State add to $I$ a shortest colorable path, not in conflict with $I$ 
\EndWhile
\\
\Return the path-recoloring corresponding to $I$


\end{algorithmic}
\caption{Greedy algorithm for 2-CR}
\label{alg:conv:greedy}
\end{algorithm}

We now describe how a shortest colorable path can be found.
%
To do that, 
at the initialization of the algorithm, 
all singletons and connected pairs should be removed from the graph.  
%
Every colorable path that is added to $I$ should be also removed from the graph.  
%
In addition, 
after each path removal, 
one should also remove all vertices with a unique color in the remaining graph.  
%
On the remaining graph, 
a shortest path, 
between two vertices of the same color, 
is guaranteed to be colorable and independent of $I$.
%
In particular,
it cannot contain two vertices that are colored by the same color, 
since such a path is not a shortest path.
