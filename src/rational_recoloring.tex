\label{sec:path_recoloring}

In this section we introduce a special type of convex recoloring,
called \emph{path-recoloring}. 
%
Path-recolorings are more constrained than general recolorings and thus are simpler to
understand and analyze.
%
Nevertheless, 
we show that there is always an optimal convex recoloring that is a path-recoloring.
%
Based on the above, 
we give an alternative definition to \TWOCR{} in terms of independent set of paths.
%
From now on, we only consider the \TWOCR{} problem,
in particular, 
whenever we mention a colored graph, 
we refer to a \TWOCR{} instance.

Given a colored graph $G_\chi$, 
if two vertices in the colored graph have the same color we call them a \emph{pair}, 
if they are connected with an edge then they are a \emph{connected pair}, 
otherwise they are a \emph{disconnected pair}.  
%
Any vertex with a unique color $c$ is a \emph{singleton}, 
we call $c$ a \emph{singleton color}.
%
We denote by $G_\chi[c]$ the subgraph induced by the set of vertices $\{v : \chi(v) = c\}$.
%
Figure~\ref{fig:concepts} depicts these concepts.


\begin{figure}[t]
\centering
\begin{tikzpicture}

\node(1) at (0, 0) [colored node, red node] {1};
\node(2) at (1, -1) [colored node, red node] {2};
\node(3) at (-1, -1) [colored node, blue node] {3};
\node(4) at (-1, 1) [colored node, blue node] {4};
\node(5) at (1, 1) [colored node, green node] {5};

\draw (1) -- (2);
\draw (1) -- (4);
\draw (1) -- (3);
\draw (1) -- (5);

\end{tikzpicture}
\caption{
	\label{fig:concepts}
	In this colored graph, vertex 5 is a singleton, vertices 1 and 2
	are a \emph{connected pair}, and vertices 3 and 4 are a disconnected pair.
}
\end{figure}

%%%%%