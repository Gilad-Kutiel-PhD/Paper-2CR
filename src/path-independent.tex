When a recoloring $\chi'$ colors a path between a disconnected pair (using the pair's color), 
we refer to the path as a \emph{colored path}.
%
Let $D$ be the set of all disconnected pairs in $G$, 
and denote by $I$ the set of colored paths in $G_{\chi'}$. 
Then the following lemma holds:

\begin{lemma}
\label{lm:cost}
Given a colored graph $G_\chi$, 
a path-recoloring $\chi'$ recolors exactly $|D| - |I|$ vertices.
\end{lemma}

\begin{proof}
By definition,
$\chi'$ may only recolor vertices from the disconnected pairs.
Also , it does not recolor any of the vertices that form the endpoints of paths in $I$.
%
$\chi'$ must recolor exactly one vertex of every other disconnected pair:
if it recolors both vertices then the color of this pair is not retained,
and if it recolors none of them then convexity does not hold.
\end{proof}

\begin{figure}
\centering
\begin{tikzpicture}
\node(1) at (-2,1) [red node] {1}; 
\node(2) at (-2,-1) [red node] {2}; 
\node(3) at (2,1) [blue node] {3}; 
\node(4) at (2,-1) [blue node] {4}; 
\node (5) at (-1,0) [green node] {5}; 
\node (6) at (1, 0) [green node] {6};
\node (7) at (0,0) [black node] {7}; 

\draw (3) -- (6); 
\draw (7) -- (6);
\draw (7) -- (5);
\draw (6) -- (4);
\draw (5) -- (2);
\draw (1) -- (5);
\end{tikzpicture}

\caption{
\label{fig:paths}
In this colored graph, the paths (4, 6, 3) and (1, 5, 2) are colorable, while
the path (6, 7, 5) is not.
Path (4, 6, 3) is in indirect conflict with path (1, 5, 2) and in direct
conflict with path (6, 7, 5).
}
\end{figure}
Given a colored graph $G_\chi$ and a path $p$, 
let $V(p)$ be the set of vertices on the path and let $\chi(p)$ 
be the set of colors assigned to vertices on this path, 
i.e. $\chi(p) = {\{\chi(v) : v \in V(p)\}}$.
%
Given two paths $p_1$ and $p_2$ in $G_\chi$:
\begin{itemize}
\item 
$p_1$ and $p_2$ are in \emph{direct conflict} if $V(p_1) \cap V(p_2) \neq \emptyset$.

\item 
$p_1$ and $p_2$ are in \emph{indirect conflict} if $\chi(p_1) \cap \chi(p_2) \neq \emptyset$,
\end{itemize}
%
$p_1$ and $p_2$ are in \emph{conflict} if they are either in a direct
or an indirect conflict 
(observe that direct conflict implies an indirect conflict).
%
If two paths are not in conflict, 
then they are \emph{independent}.
%
Given a set of paths $I$, 
we say that this set is \emph{independent}
if it is pairwise independent, 
that is, 
if every two paths $p_1, p_2 \in I$ are independent.
%
A path $u, \dots, v$ in $G$ is called \emph{colorable} if $u$ and $v$
form a disconnected pair and the path does not contain singletons nor
vertices of connected pairs.
%
Figure~\ref{fig:paths} depicts these concepts.

In the next lemma we show that the set of colored paths that is
induced by a path-recoloring is an independent set of colorable paths.

\begin{lemma}
\label{lm:independent}
Let $G_\chi$ be a colored graph.
%
Also, 
let $\chi'$ be a path-recoloring,
and let $I$ denote the set of colored paths (w.r.t.\ $\chi'$).  
%
Then $I$ is an independent set of colorable paths with respect to $\chi$.
\end{lemma}

\begin{proof}
Obviously, 
two colored paths in $I$ cannot be in direct conflict.
%
Assume for contradiction that there are two paths $p_1, p_2 \in I$
that are in indirect conflict, 
that is, 
there is a color $c \in \chi(p_1) \cap \chi(p_2)$.
%
It follows that $\chi'$ recolors two vertices of the same color, 
and we get a contradiction since $\chi'$ does not retain all colors.
%
Finally, 
it follows that all colored paths are colorable with respect to $\chi$, 
or otherwise $\chi'$ must recolor a singleton or a connected pair.
{}\end{proof}

We say that a set of paths, $I$, \emph{covers} a pair if at least one of the
vertices of the pair belongs to one of the paths in $I$.

\begin{lemma}
\label{lm:independent_set_is_recoloring}
For any independent set of colorable paths $I$ in $G_{\chi}$, 
there exists a path-recoloring $\chi'$ of $G$, 
where $I$ is the set of colored paths.
\end{lemma}

\begin{proof}
Consider a recoloring $\chi'$ that colors every colorable path in $I$
using the color of its endpoints, assigns a unique new color 
to one (arbitrary) of the endpoints of
every disconnected pair that is not covered by $I$ 
and keeps the colors of the other vertices.
%
We first show that $\chi'$ is a path recoloring.  
%
Since $I$ is independent, 
no two paths in $I$ contain the same color.
Moreover,
every path in $I$ is colorable, 
thus it does not contain singletons.
Tt follows that $\chi'$ retains all colors.  
%
Also, every path in $I$ does not contain any vertices of connected pairs, 
thus $I$ does not recolor vertices of connected pairs.
%
Finally, by the construction of $\chi'$, 
it is not hard to verify that for every color $c \in C$,
the subgraph $G_{\chi'}[c]$ contains a Hamilton path or it is a single vertex.
%
Further observe that for every color $c \in \text{Image}(\chi')$,
$G_{\chi'}[c]$ is either a simple path or a single vertex, and it
follows that $\chi'$ is a path-recoloring.  
%
Finally, by the
construction of $\chi'$, it is straightforward to verify that $I$ is
the set of colored paths induced by $\chi'$.
\end{proof}

The following is obtained due to  Lemmas~\ref{lm:cost}, 
\ref{lm:independent} and~\ref{lm:independent_set_is_recoloring}.

\begin{theorem}
\label{th:independent_set}
Given a colored graph $G_\chi$, the cost of an optimal
(path-)recoloring is $|D| - s$ if and only if the size of the maximum
independent set of colorable paths is $s$.
\end{theorem}

Theorem~\ref{th:independent_set} suggests an alternative definition to \TWOCR{}: 
given a colored graph $G_\chi$, 
find a maximum independent set of colorable paths in $G_\chi$.
